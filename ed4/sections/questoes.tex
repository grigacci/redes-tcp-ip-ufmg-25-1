\section{Questões}\label{sec:questoes}


\subsection{Questão 1}
Leia o texto sobre ATM da Seção 3.1.2 (p. 112-115, incluindo a nota
“Onde estão eles agora?” sobre ATM) e a Seção 4.3 do livro-texto e responda:


\begin{enumerate}[label=\alph*.]
    \item Explique resumidamente os princípios básicos da tecnologia ATM. Responda: por
que ATM não se tornou a tecnologia dominante em redes?
    \item O conceito básico do MPLS é o de “roteamento baseado em rótulos”. Explique
como isso funciona e quais são as vantagens.
    \item  Em que consiste o “roteamento explícito”, que vantagens ele traz e como o
MPLS pode implementá-lo?
    \item Explique como o MPLS pode ser usado para implementar uma VPN de camada 3.
\end{enumerate}

\noindent
\textbf{Resposta:} \\


\subsection{Questão 2}
 Calcule a vazão para cada um dos casos abaixo:

\begin{enumerate}[label=\alph*.]
    \item Stop and Wait, RTT = 8 ms, BW = 1 Mbps, tamanho do pacote = 1000 bytes;
    \item Go Back N, W = 2 pacotes, RTT = 8 ms, BW = 1 Mbps, tamanho do pacote = 1000
bytes;
    \item Stop and Wait, RTT = 500 ms, BW = 1 Mbps, tamanho do pacote = 1000 bytes;
    \item  Go Back N, W = 2 pacotes, RTT = 500 ms, BW = 1 Mbps, tamanho do pacote = 1000
bytes;
    \item para o cenário em (d), qual seria o tamanho mínimo de janela para que se
conseguisse atingir 100\% de utilização? 
\end{enumerate}

\noindent
\textbf{Resposta:}


\subsection{Questão 3}
Leia o artigo "End-to-End Arguments in System Design" e responda:
Muitos opositores do argumento fim a fim afirmam, entre outras coisas, que este
apenas diz que "redes devem ser o mais simples e o mais estúpidas possíveis".
Você concorda? Defensores deste argumento indicam o sucesso da Internet como
prova de que o argumento fim a fim é válido. Você concorda? Justifique suas
respostas.\\

\noindent
\textbf{Resposta:}

A ideia central, conforme o artigo original, é que funções específicas de uma aplicação devem ser implementadas nas extremidades (nos sistemas finais ou aplicações) e não nos nós intermediários da rede, a menos que haja razões de desempenho muito fortes para fazê-lo.

A lógica é que os pontos finais (as aplicações) são os únicos que têm o conhecimento completo dos requisitos da função. Tentar implementar essa funcionalidade em um nível inferior (na rede) muitas vezes resulta em uma implementação incompleta, redundante ou que não atende plenamente às necessidades da aplicação. A rede, nesse contexto, deve focar em sua tarefa primária: mover dados de um ponto a outro da forma mais eficiente e geral possível.

Desta forma, a expressão do sucesso do argumento ser a Internet é válida, pois a arquitetura da Internet foi projetada com base nesse princípio. A rede é simples e flexível, permitindo que novas aplicações sejam desenvolvidas sem a necessidade de modificações na infraestrutura da rede. Isso possibilitou uma inovação rápida e contínua, resultando em uma vasta gama de serviços e aplicações que utilizam a Internet.

Logo, a implementação de um protocolo fim a fim, se provou como uma abordagem eficaz para garantir que as aplicações possam evoluir e se adaptar às necessidades dos usuários sem depender de mudanças na infraestrutura da rede. 