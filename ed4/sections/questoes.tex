\section{Questões}\label{sec:questoes}


\subsection{Questão 1}
Leia o texto sobre ATM da Seção 3.1.2 (p. 112-115, incluindo a nota
“Onde estão eles agora?” sobre ATM) e a Seção 4.3 do livro-texto e responda:


\begin{enumerate}[label=\alph*.]
    \item Explique resumidamente os princípios básicos da tecnologia ATM. Responda: por
que ATM não se tornou a tecnologia dominante em redes?
    \item O conceito básico do MPLS é o de “roteamento baseado em rótulos”. Explique
como isso funciona e quais são as vantagens.
    \item  Em que consiste o “roteamento explícito”, que vantagens ele traz e como o
MPLS pode implementá-lo?
    \item Explique como o MPLS pode ser usado para implementar uma VPN de camada 3.
\end{enumerate}\\

\noindent
\textbf{Resposta:} \\


\subsection{Questão 2}
 Calcule a vazão para cada um dos casos abaixo:

\begin{enumerate}[label=\alph*.]
    \item Stop and Wait, RTT = 8 ms, BW = 1 Mbps, tamanho do pacote = 1000 bytes;
    \item Go Back N, W = 2 pacotes, RTT = 8 ms, BW = 1 Mbps, tamanho do pacote = 1000
bytes;
    \item Stop and Wait, RTT = 500 ms, BW = 1 Mbps, tamanho do pacote = 1000 bytes;
    \item  Go Back N, W = 2 pacotes, RTT = 500 ms, BW = 1 Mbps, tamanho do pacote = 1000
bytes;
    \item para o cenário em (d), qual seria o tamanho mínimo de janela para que se
conseguisse atingir 100\% de utilização? 
\end{enumerate}\\


\noindent
\textbf{Resposta:}


\subsection{Questão 3}
Leia o artigo "End-to-End Arguments in System Design" e responda:
Muitos opositores do argumento fim a fim afirmam, entre outras coisas, que este
apenas diz que "redes devem ser o mais simples e o mais estúpidas possíveis".
Você concorda? Defensores deste argumento indicam o sucesso da Internet como
prova de que o argumento fim a fim é válido. Você concorda? Justifique suas
respostas.\\

\noindent
\textbf{Resposta:}