\section{Questões}\label{sec:questoes}


\subsection{Questão 1}
Leia o texto sobre ATM da Seção 3.1.2 (p. 112-115, incluindo a nota
“Onde estão eles agora?” sobre ATM) e a Seção 4.3 do livro-texto e responda:


\begin{enumerate}[label=\alph*.]
    \item Explique resumidamente os princípios básicos da tecnologia ATM. Responda: por
que ATM não se tornou a tecnologia dominante em redes?
    \item O conceito básico do MPLS é o de “roteamento baseado em rótulos”. Explique
como isso funciona e quais são as vantagens.
    \item  Em que consiste o “roteamento explícito”, que vantagens ele traz e como o
MPLS pode implementá-lo?
    \item Explique como o MPLS pode ser usado para implementar uma VPN de camada 3.
\end{enumerate}

\noindent
\textbf{Resposta:} \\


\subsection{Questão 2}
 Calcule a vazão para cada um dos casos abaixo:

\begin{enumerate}[label=\alph*.]
    \item Stop and Wait, RTT = 8 ms, BW = 1 Mbps, tamanho do pacote = 1000 bytes;
    \item Go Back N, W = 2 pacotes, RTT = 8 ms, BW = 1 Mbps, tamanho do pacote = 1000
bytes;
    \item Stop and Wait, RTT = 500 ms, BW = 1 Mbps, tamanho do pacote = 1000 bytes;
    \item  Go Back N, W = 2 pacotes, RTT = 500 ms, BW = 1 Mbps, tamanho do pacote = 1000
bytes;
    \item para o cenário em (d), qual seria o tamanho mínimo de janela para que se
conseguisse atingir 100\% de utilização? 
\end{enumerate}

\noindent
\textbf{Resposta:}

Primeiro, vamos definir algumas variáveis e calcular valores comuns que serão utilizados em todos os itens:
\begin{itemize}
    \item Tamanho do pacote ($L$): 1000 bytes.
    Como 1 byte = 8 bits, então $L = 1000 \times 8 \text{ bits} = 8000 \text{ bits}$.
    \item Largura de banda (BW ou $R$): 1 Mbps.
    Como 1 Mbps = $1 \times 10^6$ bits por segundo (bps), então $R = 1 \times 10^6 \text{ bps}$.
    \item Tempo de transmissão do pacote ($T_{trans}$):
    Este é o tempo necessário para colocar todos os bits de um pacote no link.
    É calculado como $T_{trans} = \frac{L}{R}$.
    $T_{trans} = \frac{8000 \text{ bits}}{1 \times 10^6 \text{ bps}} = 0.008 \text{ segundos} = 8 \text{ milissegundos (ms)}$.
\end{itemize}
A vazão (Throughput, $Th$) é a taxa efetiva de transferência de dados. Ela pode ser calculada como o produto da Utilização do canal ($U$) pela Largura de Banda ($R$): $Th = U \times R$.

\vspace{0.5cm} % Espaçamento antes do item a

\begin{enumerate}[label=\alph*.]
    \item \textbf{Stop and Wait, RTT = 8 ms, BW = 1 Mbps, tamanho do pacote = 1000 bytes}
    \begin{itemize}
        \item Round-Trip Time (RTT) = 8 ms = 0.008 s.
        \item Tempo de transmissão ($T_{trans}$) = 8 ms = 0.008 s (calculado acima).
        \item No protocolo Stop and Wait, o transmissor envia um pacote e espera por um reconhecimento (ACK) antes de enviar o próximo. O tempo total para um ciclo completo (enviar pacote e receber ACK) é $RTT + T_{trans}$. Durante este ciclo, o transmissor está efetivamente transmitindo dados apenas durante $T_{trans}$.
        \item A utilização ($U_{SW}$) é a fração do tempo em que o transmissor está ocupado:
        $U_{SW} = \frac{T_{trans}}{RTT + T_{trans}}$
        \item Substituindo os valores:
        $U_{SW} = \frac{0.008 \text{ s}}{0.008 \text{ s} + 0.008 \text{ s}} = \frac{0.008 \text{ s}}{0.016 \text{ s}} = 0.5$
        \item A vazão ($Th_{SW}$) é:
        $Th_{SW} = U_{SW} \times R = 0.5 \times 1 \text{ Mbps} = \mathbf{0.5 \text{ Mbps}}$ (ou 500 kbps).
    \end{itemize}

    \vspace{0.3cm}

    \item \textbf{Go Back N, W = 2 pacotes, RTT = 8 ms, BW = 1 Mbps, tamanho do pacote = 1000 bytes}
    \begin{itemize}
        \item Janela de envio (W) = 2 pacotes.
        \item RTT = 8 ms = 0.008 s.
        \item $T_{trans}$ = 8 ms = 0.008 s.
        \item No Go Back N, o transmissor pode enviar até $W$ pacotes sem esperar por ACKs individuais. O canal pode ser mantido ocupado se a quantidade de dados enviados dentro da janela ($W \times L$) for suficiente para preencher o "pipe" de dados durante o tempo $RTT + T_{trans}$.
        \item A utilização ($U_{GBN}$) é dada por (considerando um cenário ideal sem perdas):
        $U_{GBN} = \min\left(1, \frac{W \times T_{trans}}{RTT + T_{trans}}\right)$
        \item Primeiro, calculamos $W \times T_{trans}$:
        $W \times T_{trans} = 2 \times 0.008 \text{ s} = 0.016 \text{ s}$.
        \item E $RTT + T_{trans}$:
        $RTT + T_{trans} = 0.008 \text{ s} + 0.008 \text{ s} = 0.016 \text{ s}$.
        \item Como $W \times T_{trans} = RTT + T_{trans}$ (ou seja, $0.016 \text{ s} = 0.016 \text{ s}$), o transmissor pode manter o canal 100\% ocupado.
        $U_{GBN} = \frac{0.016 \text{ s}}{0.016 \text{ s}} = 1$
        \item A vazão ($Th_{GBN}$) é:
        $Th_{GBN} = U_{GBN} \times R = 1 \times 1 \text{ Mbps} = \mathbf{1 \text{ Mbps}}$.
    \end{itemize}

    \vspace{0.3cm}

    \item \textbf{Stop and Wait, RTT = 500 ms, BW = 1 Mbps, tamanho do pacote = 1000 bytes}
    \begin{itemize}
        \item RTT = 500 ms = 0.5 s.
        \item $T_{trans}$ = 8 ms = 0.008 s.
        \item Usando a mesma fórmula para $U_{SW}$ do item (a):
        $U_{SW} = \frac{T_{trans}}{RTT + T_{trans}}$
        \item Substituindo os valores:
        $U_{SW} = \frac{0.008 \text{ s}}{0.5 \text{ s} + 0.008 \text{ s}} = \frac{0.008 \text{ s}}{0.508 \text{ s}} \approx 0.01574803$
        \item A vazão ($Th_{SW}$) é:
        $Th_{SW} = U_{SW} \times R \approx 0.01574803 \times 1 \text{ Mbps} \approx \mathbf{0.015748 \text{ Mbps}}$ (ou aproximadamente 15.75 kbps).
    \end{itemize}

    \vspace{0.3cm}

    \item \textbf{Go Back N, W = 2 pacotes, RTT = 500 ms, BW = 1 Mbps, tamanho do pacote = 1000 bytes}
    \begin{itemize}
        \item Janela de envio (W) = 2 pacotes.
        \item RTT = 500 ms = 0.5 s.
        \item $T_{trans}$ = 8 ms = 0.008 s.
        \item Usando a mesma fórmula para $U_{GBN}$ do item (b):
        $U_{GBN} = \min\left(1, \frac{W \times T_{trans}}{RTT + T_{trans}}\right)$
        \item Calculando $W \times T_{trans}$:
        $W \times T_{trans} = 2 \times 0.008 \text{ s} = 0.016 \text{ s}$.
        \item Calculando $RTT + T_{trans}$:
        $RTT + T_{trans} = 0.5 \text{ s} + 0.008 \text{ s} = 0.508 \text{ s}$.
        \item Como $W \times T_{trans} < RTT + T_{trans}$ (ou seja, $0.016 \text{ s} < 0.508 \text{ s}$), a janela não é grande o suficiente para manter o pipe cheio.
        $U_{GBN} = \frac{0.016 \text{ s}}{0.508 \text{ s}} \approx 0.03149606$
        \item A vazão ($Th_{GBN}$) é:
        $Th_{GBN} = U_{GBN} \times R \approx 0.03149606 \times 1 \text{ Mbps} \approx \mathbf{0.031496 \text{ Mbps}}$ (ou aproximadamente 31.50 kbps).
    \end{itemize}

    \vspace{0.3cm}

    \item \textbf{Para o cenário em (d), qual seria o tamanho mínimo de janela para que se conseguisse atingir 100\% de utilização?}
    \begin{itemize}
        \item Temos RTT = 500 ms = 0.5 s.
        \item E $T_{trans}$ = 8 ms = 0.008 s.
        \item Para atingir 100\% de utilização ($U=1$), o transmissor deve ser capaz de enviar pacotes continuamente, preenchendo o "pipe" de comunicação. O tempo total do ciclo para receber um ACK para o primeiro pacote de uma janela (e assim poder deslizar a janela) é $RTT + T_{trans}$.
        \item Durante este tempo $RTT + T_{trans}$, o número de pacotes que poderiam ter sido transmitidos se o canal estivesse continuamente ocupado é $\frac{RTT + T_{trans}}{T_{trans}}$.
        \item Para garantir que o pipe esteja sempre cheio, a janela $W$ deve ser grande o suficiente para cobrir todos esses pacotes.
        \item $W_{min} = \frac{RTT + T_{trans}}{T_{trans}}$
        \item Substituindo os valores:
        $W_{min} = \frac{0.5 \text{ s} + 0.008 \text{ s}}{0.008 \text{ s}} = \frac{0.508 \text{ s}}{0.008 \text{ s}} = 63.5$
        \item Como o tamanho da janela ($W$) deve ser um número inteiro de pacotes, e precisamos garantir que o pipe esteja sempre cheio (ou seja, que o transmissor possa enviar pelo menos $63.5$ pacotes antes de ter que parar para esperar um ACK), devemos arredondar para cima.
        \item Portanto, o tamanho mínimo da janela $W_{min}$ é $\lceil 63.5 \rceil = \mathbf{64 \text{ pacotes}}$.
        \item \textit{Verificação alternativa usando Produto Largura de Banda-Atraso (BDP):}
        O BDP é a quantidade de dados que podem estar "em trânsito" no link: $BDP = R \times RTT = 1 \times 10^6 \text{ bps} \times 0.5 \text{ s} = 500000 \text{ bits}$.
        O número de pacotes que "cabem" no pipe é $N_{pipe} = \frac{BDP}{L} = \frac{500000 \text{ bits}}{8000 \text{ bits/pacote}} = 62.5$ pacotes.
        Para manter o pipe cheio continuamente, a janela de envio $W$ deve ser grande o suficiente para cobrir esses $N_{pipe}$ pacotes mais o pacote que está sendo atualmente transmitido. Uma forma comum de expressar isso é $W \ge \frac{R \times RTT}{L} + 1$ (onde o $+1$ representa o pacote que está sendo transmitido enquanto os ACKs dos pacotes no pipe ainda não chegaram).
        $W \ge 62.5 + 1 = 63.5$. Arredondando para cima, $W_{min} = 64$ pacotes. A fórmula $W_{min} = \lceil \frac{RTT + T_{trans}}{T_{trans}} \rceil$ é mais precisa pois representa diretamente o número de "slots" de transmissão de pacote que ocorrem em um ciclo RTT + Ttrans.
    \end{itemize}
\end{enumerate}


\subsection{Questão 3}
Leia o artigo "End-to-End Arguments in System Design" e responda:
Muitos opositores do argumento fim a fim afirmam, entre outras coisas, que este
apenas diz que "redes devem ser o mais simples e o mais estúpidas possíveis".
Você concorda? Defensores deste argumento indicam o sucesso da Internet como
prova de que o argumento fim a fim é válido. Você concorda? Justifique suas
respostas.\\

\noindent
\textbf{Resposta:}

A ideia central, conforme o artigo original, é que funções específicas de uma aplicação devem ser implementadas nas extremidades (nos sistemas finais ou aplicações) e não nos nós intermediários da rede, a menos que haja razões de desempenho muito fortes para fazê-lo.

A lógica é que os pontos finais (as aplicações) são os únicos que têm o conhecimento completo dos requisitos da função. Tentar implementar essa funcionalidade em um nível inferior (na rede) muitas vezes resulta em uma implementação incompleta, redundante ou que não atende plenamente às necessidades da aplicação. A rede, nesse contexto, deve focar em sua tarefa primária: mover dados de um ponto a outro da forma mais eficiente e geral possível.

Desta forma, a expressão do sucesso do argumento ser a Internet é válida, pois a arquitetura da Internet foi projetada com base nesse princípio. A rede é simples e flexível, permitindo que novas aplicações sejam desenvolvidas sem a necessidade de modificações na infraestrutura da rede. Isso possibilitou uma inovação rápida e contínua, resultando em uma vasta gama de serviços e aplicações que utilizam a Internet.

Logo, a implementação de um protocolo fim a fim, se provou como uma abordagem eficaz para garantir que as aplicações possam evoluir e se adaptar às necessidades dos usuários sem depender de mudanças na infraestrutura da rede. 