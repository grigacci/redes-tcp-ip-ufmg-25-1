\section{Questões}\label{sec:questoes}


\subsection{Questão 1}

\textbf{Letra A.} O que vem a ser uma "RFC", dentro do contexto de documentação da Internet?
Que órgão é responsável por gerenciar estas RFCs? Qual a diferença entre uma RFC
tipo "Standards Track" e uma RFC tipo "Experimental"?
\newline
    
\textbf{Resp A.} Uma RFC (Request for Comments) é um tipo de publicação formal do IETF (Internet Engineering Task Force), usada para desenvolver e definir padrões da Internet. Elas podem conter especificações técnicas, diretrizes, ou experimentações. O órgão responsável por gerenciar essas RFCs é o próprio IETF, sob supervisão da IAB (Internet Architecture Board).
\newline 

A diferença entre uma RFC do tipo \textbf{Standards Track} e uma do tipo \textbf{Experimental} está no propósito: as primeiras seguem um processo formal de padronização e podem se tornar um padrão oficial da Internet, enquanto as experimentais propõem ideias ou tecnologias que ainda não estão maduras o suficiente para padronização. 
\newline

\textbf{Fonte:} RFC Editor – \url{https://www.rfc-editor.org/about/}, acesso em abr. 2025.
\newline

\textbf{Letra B.} Obtenha na Internet a RFC 791. Com base no texto desta RFC, detalhe o
conteúdo do campo "Type of Service" do cabeçalho do datagrama IP. 
\newline

\textbf{Resp B.} A RFC 791 descreve o protocolo IP. O campo "Type of Service" (ToS) possui 8 bits divididos em:


\begin{itemize}
    \item 3 bits para prioridade (Precedence)
    \item 1 bit para minimizar atraso
    \item 1 bit para maximizar taxa de transferência
    \item 1 bit para maximizar confiabilidade
    \item 1 reservado para uso futuro
    \item 1 não utilizado
\end{itemize}

O objetivo é fornecer diretrizes para roteadores priorizarem pacotes segundo o tipo de serviço desejado.

\textbf{Fonte:} J. Postel. \textit{RFC 791 – Internet Protocol}, setembro de 1981. Disponível em: \url{https://www.rfc-editor.org/rfc/rfc791}
\newline

\subsection{Questão 2}

Suponha que uma mensagem TCP contendo 2.048 bytes de dados e 20 bytes
de cabeçalho TCP seja passada ao IP para transmissão por duas redes da Internet
(ou seja, do host de origem, passando por um roteador até chegar ao host de
destino). A primeira rede usa cabeçalhos de 14 bytes e possui uma MTU de 1.024
bytes; a segunda usa cabeçalhos de 8 bytes com uma MTU de 512 bytes. A MTU de
cada rede indica o tamanho do maior datagrama IP que pode ser transportado em um
pacote daquela rede. Dê os tamanhos e os deslocamentos (offsets) da sequência de
fragmentos entregues à camada de rede no host de destino. Considere que todos os
cabeçalhos IP sejam de 20 bytes.
\newline

\textbf{Resp.} 
Temos 2048 bytes de dados + 20 bytes de cabeçalho TCP = 2068 bytes a serem entregues ao IP.
\newline

\textbf{Primeira rede:}
\begin{itemize}
    \item MTU = 1024 bytes
    \item Cabeçalho IP = 20 bytes $\Rightarrow$ carga útil = 1004 bytes
\end{itemize}


Fragmentos gerados:
\begin{itemize}
    \item Fragmento 1: offset 0, 1004 bytes de dados (bytes 0–1003)
    \item Fragmento 2: offset 1004, 1004 bytes (bytes 1004–2007)
    \item Fragmento 3: offset 2008, 60 bytes (bytes 2008–2067)
\end{itemize}

\textbf{Segunda rede:}
\begin{itemize}
    \item MTU = 512 bytes
    \item Cabeçalho IP = 20 bytes $\Rightarrow$ carga útil = 492 bytes
\end{itemize}

Fragmentos gerados:
\begin{itemize}
    \item Fragmento 1: offset 0, 492 bytes
    \item Fragmento 2: offset 492, 492 bytes
    \item Fragmento 3: offset 984, 492 bytes
    \item Fragmento 4: offset 1476, 492 bytes
    \item Fragmento 5: offset 1968, 100 bytes
\end{itemize}

Offsets IP são dados em unidades de 8 bytes:

\begin{itemize}
    \item Fragmento 1: offset 0  
    \item Fragmento 2: offset $\approx$ 61 (492 / 8)  
    \item Fragmento 3: offset 123  
    \item Fragmento 4: offset 185  
    \item Fragmento 5: offset 246
\end{itemize}

\textbf{Fonte:} KUROSE, J. F.; ROSS, K. W. \textit{Redes de Computadores e a Internet: Uma Abordagem Top-Down}. 7. ed. Pearson, 2018. Seção 3.2.2.
\newline

\subsection{Questão 3}
Qual é a largura de banda máxima com a qual um host IP pode enviar
pacotes de 576 bytes, sem que o campo "Ident" esgote todos os seus valores
dentro de um tempo de 60 segundos? Suponha aqui que os pacotes retardados possam
chegar com até 60 segundos de atraso, mas não mais que isso (ou serão
descartados). O que aconteceria se essa largura de banda fosse ultrapassada?
\newline

\textbf{Resp .}
O campo "Ident" tem 16 bits, ou seja, 65.536 valores possíveis. Para evitar colisões no intervalo de 60 segundos:

\[
\frac{65536~\text{pacotes}}{60~\text{s}} \approx 1092.27~\text{pacotes/s}
\]

Multiplicando pelo tamanho dos pacotes (576 bytes):

\[
1092.27 \times 576 = 628851.2~\text{bytes/s} \approx 5.03~\text{Mbps}
\]

\textbf{Conclusão:} a largura de banda máxima é cerca de 5 Mbps. Caso ultrapassada, pacotes com mesmo identificador podem ainda estar trafegando, o que causaria reassemblagem incorreta e perda de dados.

\textbf{Fonte:} Cálculo próprio baseado em especificações do IPv4 (RFC 791).
\newline

\subsection{Questão 4}
Qual foi a motivação principal para a criação de uma nova versão do
IP e por que esta nova versão demorou tanto tempo para começar a ser colocada em
operação?
\newline

\textbf{Resp .}
A motivação principal do IPv6 foi o esgotamento iminente de endereços IPv4, limitados a 32 bits. O crescimento de dispositivos conectados exigiu uma solução escalável.
\newline

A demora na adoção do IPv6 se deve à complexidade da transição, ao custo de atualização de sistemas legados e à ampla adoção de mecanismos paliativos, como NAT, que estenderam a vida útil do IPv4.
\newline

\textbf{Fonte:} KUROSE, J. F.; ROSS, K. W. \textit{Redes de Computadores e a Internet: Uma Abordagem Top-Down}. 7. ed. Pearson, 2018. Seção 4.1.3.
\newline

\subsection{Questão 5}
Quais são as principais características novas que o IPv6 traz em
relação ao IPv4?
\newline


\textbf{Resp .}
Principais inovações do IPv6 em relação ao IPv4:

\begin{enumerate}
    \item Espaço de endereços maior (128 bits)
    \item Cabeçalho simplificado
    \item Suporte nativo a IPsec
    \item Autoconfiguração (stateless)
    \item Eliminação da necessidade de NAT
    \item Melhor suporte à mobilidade e QoS
\end{enumerate}

\textbf{Fonte:} DEERING, S.; HINDEN, R. \textit{RFC 8200 – Internet Protocol, Version 6 (IPv6) Specification}, julho de 2017. \url{https://www.rfc-editor.org/rfc/rfc8200}
\newline

\subsection{Questão 6}
Explique brevemente as estratégias de pilha dupla e tunelamento,
usadas na transição do IPv4 para o IPv6.
\newline

\textbf{Resp .}
\begin{itemize}
\item \textbf{Pilha dupla (dual stack):} Permite que dispositivos operem com IPv4 e IPv6 simultaneamente, selecionando o protocolo com base no destino.

\item \textbf{Tunelamento (tunneling):} Envia pacotes IPv6 encapsulados em pacotes IPv4, permitindo que trafeguem por redes legadas sem suporte nativo a IPv6.

\end{itemize}

\textbf{Fonte:} KUROSE, J. F.; ROSS, K. W. \textit{Redes de Computadores e a Internet: Uma Abordagem Top-Down}. 7. ed. Pearson, 2018. Seção 4.1.3.
