\section{Questões}\label{sec:questoes}


\subsection{Questão 1}
Leiam as Seções 5.2.8 e 5.2.10 do livro-texto e respondam:
\begin{enumerate}[label=\alph*.]
    \item O TCP é um protocolo que vem evoluindo com o tempo. Um exemplo disso são as
extensões do TCP. Listem cada uma das extensões descritas no texto do livro,
detalhando que problema ela soluciona e como ela opera.
    \item O livro texto discute as decisões de projeto que definiram as características
do TCP e cita um outro protocolo de transporte padronizado pelo IETF, o SCTP,
que coexiste com o TCP e o UDP. Pesquisem e expliquem as diferenças do SCTP para
o TCP e por que ele não é largamente utilizado.\\
\end{enumerate}

\noindent
\textbf{Resposta:} \\

\begin{enumerate}[label=\alph*.]
    \item \begin{enumerate}[label=\roman*.]
        \item{Janela de Escala (Window Scale)}

\textbf{Problema Solucionado:} O cabeçalho TCP original possui um campo de 16 bits para o tamanho da janela de recepção, o que limita o tamanho máximo da janela a 65.535 bytes ($2^{16}$). Em redes com alto produto de largura de banda e atraso (conhecidas como \textit{long fat networks} ou LFNs), essa limitação impede o aproveitamento de toda a capacidade da rede, pois o transmissor precisa parar de enviar dados e aguardar a confirmação (ACK) com muita frequência.

\vspace{1em}

\textbf{Como Opera:} A extensão de Escala de Janela, definida na RFC 7323, introduz um fator de escala que é negociado durante o \textit{three-way handshake} da conexão TCP. Esse fator, que é um valor de deslocamento de bits, é usado para multiplicar o valor do campo de tamanho da janela. Por exemplo, um fator de escala de 7 significa que o tamanho da janela enviado no cabeçalho TCP deve ser multiplicado por $2^7$ (128). Isso permite que a janela de recepção efetiva seja muito maior que 65.535 bytes, otimizando a transmissão de dados em redes de alta performance.


\item{Marcas de Tempo (Timestamps)}

\textbf{Problema Solucionado:} Essa extensão aborda dois problemas principais: a medição imprecisa do Tempo de Ida e Volta (Round-Trip Time - RTT) e a "Numeração de Sequência Envolvida" (Wrapped Around Sequence - PAWS). O RTT é crucial para o cálculo do tempo de retransmissão, e medições imprecisas podem levar a retransmissões desnecessárias ou demoradas. Já o PAWS ocorre em redes de altíssima velocidade, onde os números de sequência de 32 bits podem se repetir rapidamente, causando confusão entre pacotes antigos e novos.

\vspace{1em}

\textbf{Como Opera:} A extensão de Marcas de Tempo adiciona um campo de 32 bits ao cabeçalho TCP. O remetente insere o seu tempo atual nesse campo. O receptor, por sua vez, ecoa esse valor de volta no campo de confirmação. Isso permite ao remetente calcular o RTT com muito mais precisão para cada pacote confirmado. Além disso, as marcas de tempo são monotonicamente crescentes, o que ajuda a distinguir pacotes com o mesmo número de sequência, mas de diferentes "eras" da conexão, resolvendo o problema do PAWS.


\item{Confirmações Seletivas (Selective Acknowledgments - SACK)}

\textbf{Problema Solucionado:} O mecanismo de confirmação original do TCP é cumulativo. Isso significa que, se um pacote for perdido, o receptor só pode confirmar a recepção dos pacotes até o ponto da perda. Consequentemente, o transmissor é forçado a retransmitir todos os pacotes a partir do segmento perdido, mesmo que alguns pacotes subsequentes tenham sido recebidos com sucesso. Isso gera retransmissões desnecessárias e ineficientes.

\vspace{1em}

\textbf{Como Opera:} A extensão SACK permite que o receptor informe ao transmissor quais blocos de dados não contíguos foram recebidos com sucesso. Isso é feito incluindo no cabeçalho TCP a informação dos números de sequência dos blocos recebidos. Com essa informação, o transmissor sabe exatamente quais segmentos de dados se perderam e pode retransmitir apenas os pacotes faltantes, tornando a recuperação de perdas muito mais eficiente. A negociação para o uso do SACK também ocorre durante o handshake inicial da conexão.


\item{Retransmissão Rápida e Recuperação Rápida (Fast Retransmit and Fast Recovery)}

\textbf{Problema Solucionado:} Esperar por um timeout (tempo de espera) para retransmitir um pacote perdido pode introduzir um atraso significativo na comunicação. A Retransmissão Rápida e a Recuperação Rápida são mecanismos que aceleram a detecção e a recuperação de perdas de pacotes.

\vspace{1em}

\textbf{Como Opera:}
\begin{itemize}
    \item \textbf{Retransmissão Rápida:} Em vez de aguardar o esgotamento de um temporizador, a Retransmissão Rápida é acionada quando o transmissor recebe um número específico de ACKs duplicados (geralmente três). Um ACK duplicado é enviado pelo receptor toda vez que ele recebe um pacote fora de ordem. A recepção de vários ACKs duplicados para o mesmo número de sequência é um forte indicativo de que o pacote seguinte foi perdido. Ao detectar isso, o transmissor retransmite imediatamente o pacote presumidamente perdido.
    
    \item \textbf{Recuperação Rápida:} Após uma Retransmissão Rápida, a Recuperação Rápida entra em ação. Em vez de reduzir drasticamente a taxa de transmissão (como no mecanismo de \textit{slow start}), este algoritmo permite que o transmissor continue a enviar novos pacotes de dados enquanto aguarda a confirmação do pacote retransmitido. Isso ajuda a manter um fluxo de dados mais constante e evita que a conexão fique ociosa, otimizando a utilização da largura de banda disponível.
\end{itemize}
    \end{enumerate}


\item O \textbf{SCTP (Stream Control Transmission Protocol)} foi projetado para combinar as melhores características do TCP (confiabilidade, controle de congestionamento) e do UDP (preservação de fronteiras de mensagem), adicionando novos recursos poderosos. No entanto, apesar de suas vantagens técnicas, ele não alcançou a mesma popularidade do TCP.

A seguir, apresentamos as principais diferenças entre os protocolos e os motivos para a baixa adoção do SCTP.

\subsection*{Diferenças: SCTP vs. TCP}

\begin{tabular}{|>{\raggedright\arraybackslash}p{3cm}|p{6cm}|p{6cm}|}
\hline
\textbf{Característica} & \textbf{SCTP (Stream Control Transmission Protocol)} & \textbf{TCP (Transmission Control Protocol)} \\
\hline
\textbf{Orientação dos Dados} & Orientado a \textbf{mensagens}. Preserva as fronteiras das mensagens, entregando-as como unidades completas. & Orientado a \textbf{bytes}. Trata os dados como um fluxo contínuo de bytes, sem noção de mensagens individuais. \\
\hline
\textbf{Múltiplos Fluxos (Multistreaming)} & \textbf{Suportado}. Permite múltiplos fluxos de dados independentes dentro de uma única conexão (associação). & \textbf{Não suportado}. Uma única conexão oferece apenas um fluxo de dados. \\
\hline
\textbf{Bloqueio "Head-of-Line"} & \textbf{Minimizado}. A perda de um pacote em um fluxo não bloqueia a entrega de pacotes nos outros fluxos. & \textbf{Problema comum}. A perda de um pacote impede a entrega de todos os pacotes subsequentes até que o perdido seja retransmitido. \\
\hline
\textbf{Múltiplas Interfaces (Multi-homing)} & \textbf{Nativo}. Permite que uma única conexão utilize múltiplos endereços IP em cada ponta, oferecendo redundância e tolerância a falhas de rede. & \textbf{Não suportado}. Uma conexão é rigidamente definida por um par de endereços IP e portas. \\
\hline
\textbf{Estabelecimento da Conexão} & \textbf{Four-Way Handshake}. Utiliza um mecanismo de "cookie" para validar o cliente antes de alocar recursos, oferecendo proteção contra ataques de inundação SYN. & \textbf{Three-Way Handshake}. Aloca recursos após o primeiro passo (pacote SYN), o que o torna vulnerável a ataques de inundação SYN. \\
\hline
\textbf{Encerramento da Conexão} & \textbf{Three-Way Handshake}. Garante que ambos os lados fechem a conexão de forma limpa, sem a possibilidade de um estado "meio-aberto". & \textbf{Four-Way Handshake}. Pode levar a um estado "meio-aberto", onde um lado fecha a conexão, mas o outro continua enviando dados. \\
\hline
\end{tabular}


\subsection*{Por que o SCTP não é Largamente Utilizado?}

Apesar de suas vantagens, o SCTP enfrenta barreiras significativas que limitaram sua adoção em larga escala na internet pública.

\begin{enumerate}
    \item \textbf{Falta de Suporte Nativo nos Sistemas Operacionais}\\
    A principal barreira é a falta de suporte "out-of-the-box" nos sistemas operacionais mais populares para desktops, como \textbf{Windows e macOS}. Embora esteja presente em sistemas como Linux e FreeBSD, a ausência em plataformas de consumo massivo torna difícil para os desenvolvedores de aplicações adotá-lo, pois isso exigiria a instalação de drivers ou bibliotecas de terceiros pelos usuários.

    \item \textbf{Problemas com a Infraestrutura de Rede Existente (Ossificação)}\\
    A internet está repleta de "middleboxes", como firewalls e dispositivos de \textbf{NAT (Network Address Translation)}, que são otimizados para TCP e UDP. Muitos desses dispositivos não reconhecem o SCTP e simplesmente bloqueiam seu tráfego. Além disso, o cálculo de checksum do SCTP (usando CRC32) é mais complexo de ser recalculado por um dispositivo NAT do que o do TCP, o que desencorajou a implementação em muitos roteadores domésticos e corporativos.

    \item \textbf{Inércia e o Ecossistema do TCP}\\
    O TCP é a base da internet há décadas. Toda a infraestrutura, ferramentas de desenvolvimento, bibliotecas de software e o conhecimento dos engenheiros de rede e desenvolvedores são construídos em torno do TCP. Migrar para um novo protocolo exigiria um esforço monumental e custos significativos, e para a maioria das aplicações (como navegação web e transferência de arquivos), o TCP é considerado "bom o suficiente".

    \item \textbf{Soluções Alternativas na Camada de Aplicação}\\
    Muitos dos problemas que o SCTP visa resolver no nível de transporte foram abordados em outras camadas. Por exemplo:
    \begin{itemize}
        \item \textbf{Multistreaming:} O \textbf{HTTP/2 e o HTTP/3 (QUIC)} implementam o conceito de múltiplos fluxos na camada de aplicação, eliminando o bloqueio "head-of-line" para aplicações web sem a necessidade de substituir o TCP/UDP.
        \item \textbf{Multi-homing:} O \textbf{Multipath TCP (MPTCP)} surgiu como uma extensão do próprio TCP para fornecer funcionalidades de multi-homing.
    \end{itemize}

    \item \textbf{Aplicações de Nicho}\\
    O SCTP encontrou seu principal caso de uso em nichos específicos, principalmente nas \textbf{redes de telecomunicações}. Ele é amplamente utilizado para transportar sinalização em redes 4G e 5G (nos protocolos Diameter e outros). Nesses ambientes controlados, as barreiras de NAT e a falta de suporte de SO não são um problema. No entanto, para a internet aberta e de uso geral, ele permanece uma solução especializada e pouco difundida.
\end{enumerate}
\end{enumerate}


\subsection{Questão 2}
 Ao fechar uma conexão TCP, por que a expiração do tempo limite de dois tempos de vida do
segmento não é necessária na transição de \texttt{LAST\_ACK} para \texttt{CLOSED}?\\


\noindent
\textbf{Resposta:}

\subsection*{O Papel do Estado \texttt{TIME\_WAIT}}
Para entender por que a transição de \texttt{LAST\_ACK} para \texttt{CLOSED} é direta, é crucial primeiro entender a função do estado \texttt{TIME\_WAIT}, que ocorre no lado que inicia o encerramento da conexão (o fechamento ativo). O temporizador de 2 MSL (aproximadamente de 1 a 4 minutos) no estado \texttt{TIME\_WAIT} tem duas finalidades críticas:

\begin{enumerate}
    \item \textbf{Garantir a Entrega do \texttt{ACK} Final:} O estado \texttt{TIME\_WAIT} garante que o último pacote \texttt{ACK} enviado pelo cliente chegue com sucesso ao servidor. Se este \texttt{ACK} for perdido, o servidor (que está no estado \texttt{LAST\_ACK}) não receberá a confirmação, seu temporizador expirará e ele reenviará seu pacote \texttt{FIN}. O cliente, ainda no estado \texttt{TIME\_WAIT}, pode então receber este \texttt{FIN} duplicado e reenviar o \texttt{ACK} final, permitindo que o servidor feche a conexão graciosamente.

    \item \textbf{Prevenir Pacotes Duplicados Atrasados:} Uma conexão TCP é definida pela tupla de 4 elementos (IP de origem, porta de origem, IP de destino, porta de destino). Após o fechamento, uma nova conexão com a mesma tupla pode ser criada. O tempo de espera de 2 MSL garante que quaisquer pacotes atrasados ("duplicatas errantes") da conexão antiga tenham tempo suficiente para expirar e serem descartados da rede, evitando que sejam erroneamente aceitos pela nova conexão.
\end{enumerate}

\subsection*{A Transição de \texttt{LAST\_ACK} para \texttt{CLOSED}}

O lado da conexão que entra no estado \texttt{LAST\_ACK} é o que realiza o fechamento passivo. Ele já recebeu um \texttt{FIN} do outro lado, respondeu com um \texttt{ACK}, e então enviou seu próprio \texttt{FIN}. Neste ponto, ele está apenas aguardando um único evento: o \texttt{ACK} final para o seu \texttt{FIN}.

A expiração do tempo limite de 2 MSL não é necessária na transição de \texttt{LAST\_ACK} para \texttt{CLOSED} pela seguinte razão:

\textbf{A responsabilidade pela robustez do encerramento é delegada ao lado em \texttt{TIME\_WAIT}.}

Uma vez que o lado em \texttt{LAST\_ACK} recebe o \texttt{ACK} final, ele tem a confirmação definitiva de que o outro lado recebeu todos os dados e o seu pedido de encerramento (\texttt{FIN}). Neste momento, ele sabe que o outro lado da conexão já fez a transição para o estado \texttt{TIME\_WAIT}.

Portanto, é o outro lado que agora assume a responsabilidade de aguardar por 2 MSL para lidar com pacotes perdidos ou duplicados. O lado em \texttt{LAST\_ACK} completou sua parte na sequência de encerramento e pode, com segurança, liberar seus recursos e transitar diretamente para o estado \texttt{CLOSED}, pois qualquer problema remanescente na rede será gerenciado pelo temporizador do estado \texttt{TIME\_WAIT} no peer. Manter um temporizador semelhante em \texttt{LAST\_ACK} seria redundante e um desperdício de recursos.



\subsection{Questão 3}
Um emissor em uma conexão TCP que recebe uma janela anunciada 0 sonda o receptor
periodicamente para descobrir quando a janela se torna diferente de zero. Por que o receptor
precisaria de um temporizador extra se ele fosse responsável por informar que sua janela
anunciada se tornou diferente de 0 (ou seja, se o transmissor não fizesse a sondagem)?\\

\noindent
\textbf{Resposta:} \\
A razão pela qual o emissor sonda o receptor em vez de esperar por uma notificação é para evitar uma condição de \textbf{deadlock} (impasse). 

Se o receptor fosse o único responsável por anunciar uma nova janela (diferente de zero), um problema crítico surgiria se esse pacote de anúncio de janela fosse perdido na rede. Nesse cenário:
\begin{enumerate}
    \item O \textbf{emissor} recebeu uma janela de 0 e parou de enviar dados. Ele ficaria esperando passivamente por um anúncio de que a janela foi reaberta. 
    \item O \textbf{receptor}, por sua vez, já teria enviado o anúncio de reabertura da janela e estaria esperando a chegada de novos dados do emissor. 
\end{enumerate}

Como o anúncio do receptor se perdeu, o emissor nunca saberia que pode voltar a transmitir. Como o emissor não pode enviar novos dados (sua janela é 0), ele não pode acionar nenhuma resposta (ACK) do receptor. Ambas as partes ficariam esperando indefinidamente uma pela outra, resultando em um deadlock.

Para que o receptor pudesse quebrar esse impasse, ele precisaria de um \textbf{temporizador extra}. Este temporizador seria acionado após o envio de um anúncio de reabertura de janela. Se nenhum dado novo chegasse do emissor antes que o temporizador expirasse, o receptor teria que presumir que seu anúncio se perdeu e, então, reenviá-lo.

O TCP evita essa complexidade adicional no receptor seguindo o princípio de projeto de "emissor inteligente/receptor burro" (\textit{smart sender/dumb receiver}).  A responsabilidade de quebrar o impasse é colocada no emissor, que já gerencia temporizadores de retransmissão. Ao sondar periodicamente o receptor com um pequeno pacote, o emissor garante que receberá uma resposta com o tamanho da janela atual, quebrando o potencial deadlock de forma robusta e mantendo o design do receptor mais simples. 

\subsection{Questão 4}
O campo de número de sequência no cabeçalho TCP tem 32 bits de extensão, que é grande o
suficiente para cobrir mais de 4 bilhões de dados. Mesmo que todos esses bytes nunca sejam
transferidos em uma única conexão, por que o número de sequência ainda pode se reiniciar
ciclicamente de $2^{32}-1$ para 0?\\

\noindent
\textbf{Resposta:} \\
O número de sequência do TCP pode reiniciar ciclicamente (dar a volta, ou "wrap around") de $2^{32}-1$ para 0 muito antes de 4 GiB de dados serem transferidos porque o \textbf{Número de Sequência Inicial (ISN - Initial Sequence Number) não começa em 0}. 

A especificação do TCP exige que cada lado de uma conexão selecione um ISN \textbf{aleatoriamente} a partir do espaço de 32 bits.  O motivo para essa aleatoriedade é uma medida de segurança: proteger a conexão contra pacotes antigos e duplicados de uma instância anterior da mesma conexão (mesmo par de IPs e portas) que ainda possam estar "vagando" pela rede.  Se um ISN previsível fosse usado, esses pacotes antigos poderiam ser erroneamente aceitos pela nova conexão.

Como o ISN é um número aleatório de 32 bits, ele pode, por acaso, ser um valor muito alto, próximo do limite de $2^{32}-1$. Por exemplo, se o ISN escolhido para uma conexão for $2^{32}-1000$, então, após a transferência de apenas 1001 bytes, o número de sequência irá ultrapassar $2^{32}-1$ e reiniciar em 0.

Portanto, o reinício cíclico não está ligado à quantidade total de dados transferidos desde o byte 0, mas sim à quantidade de dados transferidos desde o ISN aleatório escolhido no início da conexão.

\noindent
\textbf{Resposta:}

\subsection{Questão 5}
Você foi encarregado de projetar um protocolo de fluxo de bytes confiável que use janela
deslizante (como o TCP). Esse protocolo será executado em uma rede de 100 Mbps. O RTT da
rede é de 100 ms, e o tempo de vida máximo dos segmentos é de 30 segundos.
Quantos bits você incluiria nos campos JanelaAnunciada e NúmeroSeq do cabeçalho do seu
protocolo?\\

\noindent
\textbf{Resposta:}

\subsection{Questão 6}
Quando o TCP envia um \texttt{<SYN, NúmeroSeq=x>} ou \texttt{<FIN, NúmeroSeq=x>}, o ACK correspondente
possui Confirmação=x+1; ou seja, SYNs e FINs ocupam uma unidade no espaço do número
de sequência. Isso é necessário? Se for, dê um exemplo de uma ambiguidade que surgiria se a
Confirmação correspondente fosse x em vez de x+1; se não, explique por quê.\\

\noindent
\textbf{Resposta:}