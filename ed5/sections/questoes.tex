\section{Questões}\label{sec:questoes}


\subsection{Questão 1}
Leiam as Seções 5.2.8 e 5.2.10 do livro-texto e respondam:
\begin{enumerate}[label=\alph*.]
    \item O TCP é um protocolo que vem evoluindo com o tempo. Um exemplo disso são as
extensões do TCP. Listem cada uma das extensões descritas no texto do livro,
detalhando que problema ela soluciona e como ela opera.
    \item O livro texto discute as decisões de projeto que definiram as características
do TCP e cita um outro protocolo de transporte padronizado pelo IETF, o SCTP,
que coexiste com o TCP e o UDP. Pesquisem e expliquem as diferenças do SCTP para
o TCP e por que ele não é largamente utilizado.
\end{enumerate}\\



\noindent
\textbf{Resposta:} \\


\subsection{Questão 2}
 Ao fechar uma conexão TCP, por que a expiração do tempo limite de dois tempos de vida do
segmento não é necessária na transição de LAST_ACK para CLOSED?\\


\noindent
\textbf{Resposta:}


\subsection{Questão 3}
Um emissor em uma conexão TCP que recebe uma janela anunciada 0 sonda o receptor
periodicamente para descobrir quando a janela se torna diferente de zero. Por que o receptor
precisaria de um temporizador extra se ele fosse responsável por informar que sua janela
anunciada se tornou diferente de 0 (ou seja, se o transmissor não fizesse a sondagem)?\\

\noindent
\textbf{Resposta:}

\subsection{Questão 4}
O campo de número de sequência no cabeçalho TCP tem 32 bits de extensão, que é grande o
suficiente para cobrir mais de 4 bilhões de dados. Mesmo que todos esses bytes nunca sejam
transferidos em uma única conexão, por que o número de sequência ainda pode se reiniciar
ciclicamente de 2^{32}-1 para 0?\\

\noindent
\textbf{Resposta:}

\subsection{Questão 5}
Você foi encarregado de projetar um protocolo de fluxo de bytes confiável que use janela
deslizante (como o TCP). Esse protocolo será executado em uma rede de 100 Mbps. O RTT da
rede é de 100 ms, e o tempo de vida máximo dos segmentos é de 30 segundos.
Quantos bits você incluiria nos campos JanelaAnunciada e NúmeroSeq do cabeçalho do seu
protocolo?\\

\noindent
\textbf{Resposta:}

\subsection{Questão 6}
Quando o TCP envia um 〈SYN, NúmeroSeq=x〉 ou 〈FIN, NúmeroSeq=x〉, o ACK correspondente
possui Confirmação=x+1; ou seja, SYNs e FINs ocupam uma unidade no espaço do número
de sequência. Isso é necessário? Se for, dê um exemplo de uma ambiguidade que surgiria se a
Confirmação correspondente fosse x em vez de x+1; se não, explique por quê.\\

\noindent
\textbf{Resposta:}